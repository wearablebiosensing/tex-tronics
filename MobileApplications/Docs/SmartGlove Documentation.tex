%Copyright 2014 Jean-Philippe Eisenbarth
%This program is free software: you can 
%redistribute it and/or modify it under the terms of the GNU General Public 
%License as published by the Free Software Foundation, either version 3 of the 
%License, or (at your option) any later version.
%This program is distributed in the hope that it will be useful,but WITHOUT ANY 
%WARRANTY; without even the implied warranty of MERCHANTABILITY or FITNESS FOR A 
%PARTICULAR PURPOSE. See the GNU General Public License for more details.
%You should have received a copy of the GNU General Public License along with 
%this program.  If not, see <http://www.gnu.org/licenses/>.

%Based on the code of Yiannis Lazarides
%http://tex.stackexchange.com/questions/42602/software-requirements-specification-with-latex
%http://tex.stackexchange.com/users/963/yiannis-lazarides
%Also based on the template of Karl E. Wiegers
%http://www.se.rit.edu/~emad/teaching/slides/srs_template_sep14.pdf
%http://karlwiegers.com
\documentclass{scrreprt}
\usepackage{listings}
\usepackage{underscore}
\usepackage[bookmarks=true]{hyperref}
\usepackage[utf8]{inputenc}
\usepackage[english]{babel}
\hypersetup{
    bookmarks=false,    % show bookmarks bar?
    pdftitle={SmartGlove Application User's Guide},    % title
    pdfauthor={Matthew Constant},                     % author
    pdfsubject={SmartGlove Application Development},                        % subject of the document
    pdfkeywords={SmartGlove, API, Android}, % list of keywords
    colorlinks=true,       % false: boxed links; true: colored links
    linkcolor=blue,       % color of internal links
    citecolor=black,       % color of links to bibliography
    filecolor=black,        % color of file links
    urlcolor=purple,        % color of external links
    linktoc=page            % only page is linked
}%
\def\myversion{1.0 }
\date{}
%\title
\usepackage{hyperref}
\begin{document}

\begin{flushright}
    \rule{16cm}{5pt}\vskip1cm
    \begin{bfseries}
        \Huge{User's Guide}\\
        \vspace{1.9cm}
        for\\
        \vspace{1.9cm}
        SmartGlove Android Application\\
        \vspace{1.9cm}
        \LARGE{Version \myversion}\\
        \vspace{1.9cm}
        Prepared by Matthew Constant\\
        \vspace{1.9cm}
       Wearable Biosensing Lab\\
        \vspace{1.9cm}
        \today\\
    \end{bfseries}
\end{flushright}

\tableofcontents


%\chapter*{Revision History}

%\begin{center}
    %\begin{tabular}{|c|c|c|c|}
        %\hline
	 %   Name & Date & Reason For Changes & Version\\
        %\hline
	 %   21 & 22 & 23 & 24\\
%        \hline
%	    31 & 32 & 33 & 34\\
    %    \hline
   % \end{tabular}
%\end{center}

\chapter{Introduction}

\section{Purpose}
This application is being developed in order to act as a gateway between the SmartGlove
e-Textile device and the secure server on which the data collected is to be stored and
viewed from. By collecting this data from the SmartGlove device in real-time, the application
has the ability to both guide the participant through his/her exercise as well as securely store
this data in a server to be accessed at a later time by either the doctor or patient. In 
addition, this application will allow users to easily visualize the data that is stored on the 
server in a simple and intuitive way.

\section{Application Requirements}
This application must connect to the SmartGlove device with as little user interaction
as possible. This includes filtering devices out when scanning for nearby BLE devices,
as well as automatically reconnecting to previously connected devices. This application
must also be able to send the collected data to a server via MQTT. Lastly, the application
needs to be able to connect to the remote server in order to retreive and visualize 
a patient's data in the application.

\section{Implementation}
This application has implemented these requirements using the native Android BLE API
and the open source MPAndroid Charts library. (The MQTT portion of this application
is still under development). Developers can interact with the SmartGlove using the 
BleConnectionService class. This class allows users the ability to connect to,
disconnect from, discover services, read, write, and be notified. The data retrieved
is then broadcast using the BleUpdateReceiver. This means that any class listening 
via a BleUpdateReceiver can collect information from the BleConnectionService. 
This approach allows for the highest level of abstraction as well as the least impactful
on the User's program. All actions supported by the BleConnectionService are made 
available by static methods in the BleConnectionService. Another layer of abstraction
is made by using the BluetoothLeModel class to represent the BLE device to the user.
The user will create a BluetoothLeModel and use this to send and receive data from the
BleConnectionService.

\section{Resources}



\chapter{Overview}

\section{BluetoothLeModel}
Rather than interact directly with the BluetoothGatt class, the user will only need to
use the BluetoothLeModel. This is a much more simplified class that still allows the 
User to take full advantage of the SmartGlove device. The BluetoothLeModel
contains the BLE Device address, as well as all of its services and characteristics.
For example, in order to connect to a nearby BLE device the user can simpy
create a BluetoothLeModel with its Bluetooth Address using the following 
method: BluetoothLeModel.CREATE(String BT_ADDR). The BluetoothLeModel
can allows be created with all of its services and characteristics by first discovering the
devices services and adding this list to the create method. For example:\\
$BleConnectionService.DISCOVER_SERVICES(Context, BluetoothLeModel)\\
...\\
BluetoothLeModel.CREATE(String BT_ADDR, List<BluetoothGattServices>);$\\

\section{Sending Commands to BleConnectionService}
Users can interact with the SmartGlove by sending \''Actions\'' to the
BleConnectionService. These actions are made available with static methods.
For example, in order to connect to a nearby BLE device, the user can simply
call BleConnectionService(Context, BluetoothLeModel). The following commands
are currently implemented by the BleConnectionService:\\
CONNECT\\
DISCONNECT\\
DISCOVER_SERVICES\\
REQUEST_READ\\

\section{Receiving Updates and Data from BleConnectionService}
In order to retrieve information from the BleConnectionService, the user must implement
the BleUpdateReceiver. All data and updates from the BleConnectionService are transmitted
through this Receiver. Updates includes when a device is connected to or disconnected from,
the services of a device are discovered, and when the device is read from or written to.
Each update will include the corresponding BluetoothLeModel, providing the user
with the Device address and the most current values for each characteristic.

\section{Adding Features to BleConnectionService}
Implementing new features to the BleConnectionService is a very straight-forward process.
First, the action must be defined in the BleActions enum. Next, a static method must be implemented
in the BleConnectionService to provide an interface for users to use this feature (Please follow the 
pattern of all other features, meaning the static method should simply generate the intent needed and
send that intent to the service). Third, add the action into the switch statement in the onStartCommand
method in BleConnectionService. This should do some preliminary error checking and then call the corresponding
method related to this action. The last step is to implement this method that is related to the action. If this
feature must also send data or updates back, it must be sent using the BleUpdate Broadcast Receiver.

\section{Areas to Improve}



\end{document}